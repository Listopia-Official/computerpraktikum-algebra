\documentclass[9pt]{beamer}
\usepackage[sfdefault]{roboto}
\usepackage[utf8]{inputenc}
\usepackage[T1]{fontenc}
\usepackage{styles/fluxmacros} 	% Define where theme files are located. 
\usefolder{styles}
\usetheme[style=red]{flux} % Available styles: asphalt, blue, red, green, gray 

\title{Computerpraktikum Algebra}
\subtitle{Thema 4 - Graphen und Lie-Algebren}
\author{Pascal Bauer, Raphael Millon, Florian Haas}
\institute{Sommersemester 2020}
\date{\today}
\titlegraphic{assets/overleaf.png}

\begin{document}

\titlepage 

\begin{frame}
 \frametitle{Table of contents}
 \tableofcontents
\end{frame}

\section{Kurzer Ausflug in die Theorie}
\begin{frame}{Flux}{Theorie}
\centering
	Theorie\\
	trocken\\
	diese Dürre\\
	sie trocknet alles aus\\
	die Theorie\\
	entzieht das Wasser\\
	und es bleibt\\
	Staub.\\[0.5em]
\small\textit{Frei nach Shian Olaraf}
\end{frame}
\section{Showcase}
\begin{frame}{Flux}{Showcase}
gmat\\
glin\\
gphi\\
\end{frame}
\section{Ausgesuchte Codebeispiele}
\begin{frame}{Flux}{Codebeispiele}
GAP GAP GAP
\end{frame}
\end{document}